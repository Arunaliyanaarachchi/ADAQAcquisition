\chapter{Tutorial 1: Creating a calibrated energy spectrum}
\label{chap:tutorial1}
This chapter provides a detailed, step-by-step tutorial that covers
using \ADAQ to create a calibrated energy spectrum and output the
resulting spectrum to an encapsulated post script (EPS) file.
Specifically, the user will learn how to establish a VME connection to
the V6534 and V1720 boards, power a standard sodium-iodide detector
with the V6534 board, and acquire digital detector waveforms to create
an energy spectrum.

\section{Assumptions}
\label{sec:assumptions1}
For this walkthrough, a number of \textit{a priori} assumptions have
been made:
\begin{itemize}
  \item{The CAEN V1718 driver has been installed on system running
    \texttt{CyDAQRootGUI}.}
  \item{The user's environment is configured as described in
    Section~\ref{sec:config}.}
  \item{The CAEN V1718 board occupies slot 1 of the VME8100 crate; the
    CAEN V6534 and V1720 VME boards have their VME address set to
    0x42420000 and 0x00420000, respectively, using the physical
    potentiometers on each board.}
  \item{The system running \texttt{CyDAQRootGUI} is connected to the
    V1718 board via the USB-2.0 cable.}
  \item{The NaI-PMT detector requires 1750 volts of \textit{negative
      bias} bias high voltage and less than 800 microamps of
    current.}
  \item{The NaI detector outputs negative pulses (i.e. pulses that
    ``rise'' below the baseline in the range -1 to +1 volts. Typical
    pulses are on the order of a few hundred millivolts to a
    thousand millivolts.}
  \item{The user has a $^{137}$Cs (E$_{\gamma}$ = 661.7 keV) and
    $^{60}$Co (E$_{\gamma}$ = 1170 and 1330 keV) radioactive
    source available.}
\end{itemize}


\section{Hardware setup}
\label{sec:hardwaresetup1}
This section describes the ordered steps for setting up the \ADAQ
hardware in advance of running \texttt{CyDAQRootGUI}.

\begin{enumerate}
\item{Power on the VME8100 crate.}
\item{Connect the V6534 high voltage channel 0 (SHV connector
  required) to the high voltage input on the base of the NaI
  detector.}
\item{Connect the V1720 digitizer channel 0 (MCX jack required) to
  the signal output on the base of the NaI detector.}
\end{enumerate}

\section{Configure \ADAQ for acquisition}
\label{sec:configure1}
This section describes starting the \texttt{CyDAQRootGUI} software,
establishing a VME connection to the CAEN V6534 and V1720 boards, and
powering the NaI detector using the V6534 high voltage board.

\begin{enumerate}
  \item{Open a terminal and launch the \ADAQ graphical interface by
    typing ``CyDAQRootGUI'' at the command line. You should see the
    graphical interface open to the ``VME Connection'' frame described
    in Section~\ref{sec:vmeconnectionframe}.}
  \item{Ensure that the V6534 and V1720 VME base addresses are
    correctly set (0x4242000 for the V6534 high voltage board and
    0x00420000 in this example).}
  \item{Establish a VME connection to the V6534 and V1720 boards by
    clicking the large, red connect button at the top of VME
    connection frame. If the connection is successfully established,
    the button will turn green and the button's text will indicate a
    successful connection.}
  \item{Use the tabs at the top-left of the GUI and navigate to the
    ``High Voltage'' frame. Using the ``Set'' widgets for V6534
    channel 0, assign a value of ``1750'' (volts) to the ``Set
    Voltage'' number entry and a value 'f ``800'' (microamps) to the
    ``Set Current'' number entry. Click the ``Enable monitoring''
    check box at the bottom of the screen; all 6 channel's ``Active
    Voltage'' and ``Active Current'' widgets will turn white, showing
    they are enabled. Each widget's value should be 0 (or very close
    to it) since none of the V6534 high voltage channels are on.}
  \item{Click the large, red ``Power button'' at the rightmost side of
    channel 0's group frame. The button will turn from red to green
    and the text will change from ``OFF'' to ``ON'' to indicate that
    V6534 channel 0 is now energized. You should now observe the
    voltage and current on channel 0 slowly ramp up by obvserving the
    ``Active Voltage'' and ``Active Current'' widgets for channel
    0. Note that while the active voltage will rise to the set value
    of 1750 volts, the active current will rise to whatever current is
    drawn by the NaI detector's PMT and base electronics.}
\end{enumerate}

\section{Acquiring waveforms}
\label{sec:acquiriing1}
This section describes using \texttt{CyDAQRootGUI} to optimally
acquire digital waveforms from the detector output pulsest At this
point in the tutorial, the \ADAQ system hardware should be producing
analog output NaI detector pulses which are being fed into the V1720
digitizer channel 0.

\begin{enumerate}
  \item{Use the tabs at the top-left of the GUI and navigate to the
    ``Oscilloscope'' frame. In the channel 0 group frame, ensure that
    digitizer channel 0 is enabled. Use the rest of the channel 0
    default settings.}
  \item{Under the ``Scope Settings'' subframe in the ``Trigger'' group
    frame, set the trigger type to ``Software''.}
  \item{Click the wide, red button under the blank DGScope canvas. The
    button will turn from red to green and the text will change from
    ``Stopped'' to ``Acquiring'' to indicate that the V1720 board is
    now running and waiting to acquire data.}
  \item{Under the ``Trigger'' group frame, click the ``Manual
    Trigger'' button to send a software trigger to the V1720
    board. This will tell the V1720 board to convert any analog signal
    on enabled channels (only 0 in this case) to a digital
    waveform. You should see an oscilloscope-like plot appear on the
    DGScope canvas. If the data acquisition gods are on your side, you
    should see a horizontal solid black line and a horizontal dashed
    black line appear somwhere in the middle of the plot. The solid
    line is the baseline for channel 0; the dashed line is the trigger
    threshold for channel 0. Note that the trigger threshold's y-value
    on the plot corresponds to the trigger threshold number entry
    setting in the channel 0 group frame.}
  \item{Using the ``Trig. Threshold'' number entry in the channel 0
    group frame, Set the trigger threshold to be 100-300 ADC units
    below the channel 0 baseline (because the NaI detector is assumed
    to have negative pulses polarity). Clicking the ``Manual Trigger''
    button will update the position of the channel 0 trigger threshold
    line on the DGScope canvas.}
  \item{Click the wide, green acquisition button below the DGScope
    canvas to cease data acquisition. Under the trigger groupframe,
    change the trigger type to ``Automatic'' using the selection
    box. This tells the V1720 board to trigger channel 0 each time it
    senses a digitized value that exceeds the channel 0 trigger
    threshold that was just set.}
  \item{Click the wide, red button under the DGScope canvas to start
    data acquisition. If, for some obscure reason, the data
    acquisition gods continue to favor your mere mortal self, you
    should now see frequent waveforms appearing on the DGScope. Note
    that only waveforms whose height exceeds the trigger threshold
    appear. You may adjust the trigger threshold in real time to see
    changes. Note that high acquisition rates cause a lag between the
    acquired pulse (relatively low CPU) and plotting the waveform
    (relatively high CPU). A quick ``off/on'' cycle of acquisition
    will usually help when wishing to view the effects of live-time
    trigger changes.}
  \item{Play around with the other settings under the channel 0 group
    frame. Note that the min/max baseline settings result in a shaded
    box drawn over the waveform baseline. This region is used to
    calculate the channel baseline and must not overlap the pulse in
    the acquisition window!}
  \item{Play around with the vertical double and horizontal triple
    sliders that are to the left and bottom, respectively, of the
    DGScope. Using the vertical slider, zoom in on the channel
    baseline to a scale of a few tens of ADC. Note the inevitable
    presence of noise! Using the horizontal slider, magnify certain
    regions of the pulse by changing the fraction of the total
    acquisition window that is viewd in the DGScope canvas.}
  \item{Stop the acquisition by clicking the wide, green button.}
\end{enumerate}

\section{Creating a calibrated  energy spectrum}
This section describes creating adjusting the \texttt{CyDAQRootGUI}
settings to successfully create pulse spectra (``uncalibrated'') and
energy spectra (``calibrated'').

\begin{enumerate}
  \item{Start waveform acquisition again by clicking the wide, red
    button. Using the widgets in the channel 0 group frame, ensure the
    trigger threshold is near (but not too close) to the baseline, i.e
    above all noise by at least 10 or 20 ADC. Ensure that no
    saturation of waveforms occurs (except for maybe the occasional
    large cosmic ray). Ensure that the waveform is comfortably within
    the acquisition window (if not, change the number of samples in
    the acquisition window using the ``Record Length'' number entry or
    adjust the number of samples that occur after the trigger using
    the ``Post trigger'' number entry in the acquisition group
    frame). Ensure that the shaded baseline calculation region does
    not overlap the pulses}
  \item{Stop the acquisition.}
  \item{Change DGScope to run in ``spectrum'' mode by clicking the
    appropriate radio button in the ``DGScope Mode'' group frame.}
  \item{Click the ``Spectrum Settings'' subtab to display the options
    that effects the spectrum histogram. Leave the settings at their
    default value. In this example we will create a ``pulse area
    spectrum'', where the entire detector waveform above the baseline
    in the triggered acquisition window is integrated and the
    final value histogrammed.}
    \item{Start the acquisition}
    \item{After sufficient detector waveforms have been acquired,
      integrated, and histogrammed, the DGScope canvas should begin to
      display the histogrammed pulse area spectrum. By default, the
      spectrum histogram painted in the DGScope canvas is updated
      every 100 events.}
    \item{By default, the spectrum histgram has 100 bins evenly
      divided between 0 and 30000 pulse units. If the histogram is cut
      off at higher energy, stop the acquisition, extend the histogram
      range by increasing the ``Maximum bin'' value on in the
      ``Spectrum Settings'' subframe to the desired level (possibly
      increasing the number of bins as well), and restart the
      acquisition. Repeat until the full expected range of the pulse
      spectrum is within the bin limits of the spectrum histogram.}
    \item{Expose the detector to the $^{137}$Cs source, and ensure
      that the spectrum contains the expectd features for a NaI
      detector: photoabsorption peak, Compton gap and edge, Compton
      continuum, gamma backscatter peak, etc.}
    \item{Click the ``Calibration Mode'' check box on the right side
      of the ``Spectrum Settings'' subframe. A vertical dotted red
      line should appear in the center of the spectrum in the DGScope
      canvas. Grab the little slider on the horizontal triple slider
      just below the DGScope canvas and drag it back and forth. Notice
      how the vertical red line moves with the spectrum and the
      ``Pulse Unit'' number entry updates according to the position of
      the vertical red line in the spectrum.}
    \item{Place the $^{137}$Cs source next to the NaI detector until
      the photopeak corresponding to the 661.7 keV gamma appears in
      the pulse area spectrum histogram in the DGScope canvas. Drag
      the little slider to align the vertical red line with the center
      of the photopeak. Enter ``661.7'' in the ``Calibration Energy''
      number entry. Click the ``Set Point'' button when finished.}
    \item{Remove the $^{137}$ Cs source and place the $^{60}$Co source
      nera the NaI detector; wait for the photopeaks to appear in the
      pulse area spectrum corresponding to the 1170 and 1330 keV
      gammas. Align the red vertical line with the 1170 keV photopeak,
      enter ``1170'' in the ``Calibration Energy'' number entry, and
      click `the ``Set Point'' button. Repeat for the 1330 photopeak.}
    \item{remove the $^{60}$Co source and click the ``Calibrate''
      button and uncheck the ``Calibration Mode'' button.}
    \item{At this point, the values of the histogram bins are in units
      of keV and the histogram bin limits should be reset to reflect
      the energy of the incident gammas. Stop the acquisition, change
      the ``Maximum bin'' number entry to ``1600'', and restart the
      acquisition. The histogram will now bin detector waveforms
      between 0 and 1600 keV. Expose the NaI detector to the
      $^{137}$Cs and $^{60}$Co source, verifying that the three gamma
      photopeaks appear at the extected energy value in the spectrum
      histogram (661.7, 1170, 1330 keV). Stop the acquisition.}
\end{enumerate}

\section{Creating an EPS file of the energy spectrum}
This section describes outputting the contents of the DGScope canvas,
which will contained a labelled gamma energy spectrum, to an EPS file.

\begin{enumerate}
  \item{Using the NaI detector and the $^{137}$Cs and $^{60}$Co
    sources, set the desired spectrum limits and number of bins, begin
    acquisition, and create an energy spectrum. Leave the acquisition
    running.}
  \item{Click the ``Graphical Options'' tab to move the the DGScope
    graphical settings frame. Set the X-axis, Y-axis, and graph titles
    using the appropriate text entries. Adjust the position of the
    X-axis and Y-axis positions as needed.}
  \item{Set the output image file name using the text entry at the
    right of the ``Graphical Options'' subframe. Note that the file
    suffix (.eps, .ps, etc) is automatically attached to the file
    name. Use the selection box underneath to ensure that the ``.eps''
    options is selected to ensure the output file is an encapsulated
    poscript.}
  \item{Click the ``Save plot'' button, which will create an EPS file
    containing the current contents of the DGScope canvas. If the file
    name ``EnergySpectrum'' was used, a file name
    ``EnergySpectrum0.eps'' will be created in the directory from
    whence \texttt{CyDAQRootGUI} was executed. Use your favorite
    postscript viewer to inspect the contents of the file.}
  \item{Click the ``Save plot'' button again, with the same file name
    and file type. If the file name ``EnergySpectrum'' was used, a
    file named ``EnergySpectrum1.eps'' will be created in the
    directory from whence \texttt{CyDAQRootGUI} was started. The
    incrementing integer ensures images with the same file name are
    not overwritten!}
\end{enumerate}

\section{Conclusion}
Congratulations! If you have reached this point without throwing
errors or putting your first through the computer screen, you have
successfully used \ADAQ to acquire detector waveform data, analyze it to
create a calibrated energy spectrum, and saved it into a
vector-graphics EPS file, laballed axis and all. The interested user
is recommended to repeat this tutorial but explore the numerous
\texttt{CyDAQRootGUI} settings that were not used in this tutorial.

