\chapter{Introduction to ADAQAcquisition}
\label{chap:introdurction}

\ADAQ stands for \textbf{AIMS} \textbf{D}ata
\textbf{A}c\textbf{Q}uisition. In general, \ADAQ is a collection of
software tools running under Linux\footnote{Porting of \ADAQ to
  Windows/Mac platforms is not foreseen at this time} for digitizing
and analyzing output waveforms from the particle detectors that will
be used with Ionetix cyclotron experiments. \ADAQ has been explicitly
developed for use with \CAEN data acquisition hardware and the \CAEN C
libraries that are distributed in support of the various hardware
modules. At present, support is provided for a single \CAEN V6534 high
voltage VME board and a signle \CAEN V1720 digitizerVME board
controlled via the \CAEN V1718 USB-VME board.

\section{Overview of the \ADAQ software}
\label{sec: overviewsoftware}
The centerpiece of \ADAQ is the \texttt{ADAQAcquisitionRootGUI} program, which
provides an extremely powerful graphical user interface (GUI) for full
control of the V6534 and 1720 boards, as well greatly enhanced
\textit{digital versions} of the oscilloscope (waveform analysis),
multichannel analyzer (pulse spectra creation) and computer programs
(graphical plotting, digitial data storage) found in most traditional
analog DAQ setups.

\texttt{ADAQAcquisitionRootGUI} is primarily bnuilt upon two families of C++
classes. First, \texttt{ADAQAcquisitionRootGUI} integrates many classes from the
\ROOT data analysis toolkit that not only provides the graphical
interface libraries themselves but also a tremendously powerful set of
data analysis tools, such as graphing, histogramming, curve fitting,
peak finding algorithms and much, much more. Second,
\texttt{ADAQAcquisitionRootGUI} uses two custom C++ classes (ADAQAcquisitionHigh voltage
and ADAQAcquisitionDigitizer) that provide a simple but powerful way for the \ADAQ
developer to control the V6534 and V1720 boards, obscuring the
nitty-gritty of interacting with the boards via the CAEN libraries and
exposing easy-to-understand and often significantly enhanced methods
for controlling the boards.

Another important component of \ADAQ are the offline waveform analysis
tools, which again make full use of the \ROOT toolkit. At present, a
skeleton template code is provided that will open a specified data
file containing waveforms, read in all the relevant parameters from
the data file, and then iterate through all the stored waveforms in
the file. The user is expected to use this template to implement
his/her own algorithsm for waveform analysis (counting, pulse spectra
creation, pulse shape discrimination, etc).

The purpose of this user's guide is to provide the user of \ADAQ with
the information required to
\begin{enumerate}
  \item{Understand the general layout of the \ADAQ source code}
  \item{Retrieve the \ADAQ source code from the \GIT\footnote{Subversion
      code management system. Available at:
      \purl{http://subversion.apache.org}} repository and build the
    \texttt{ADAQAcquisitionRootGUI} executable from source.}
  \item{use \texttt{ADAQAcquisitionRootGUI} to acquire and analyze output
    detector waveforms}
  \item{Begin developing customized offline analysis code to process
    digital waveform data}
\end{enumerate}

To achieve these goals, the remainder of this chapter describes the
external software dependencies required to build the \ADAQ source code
and run its programs. Chapter~\ref{chap:svn} provides instructions on
how to obtain the \ADAQ source from the \GIT repository and building
\texttt{ADAQAcquisitionRootGUI} from source, including configuring the user's
Linux environment and some notes on obtaining and building the
external software dependencies. Chapter~\ref{chap:overview} provides a
detailed guide to the \texttt{ADAQAcquisitionRootGUI} program, including full
textual descriptions of each graphical widget available to the user as
well as annotated graphics of various parts of the
program. Chapters~\ref{chap:tutorial1} and \ref{chap:tutorial2}
contain step-by-step tutorials that walk the user through some of the
most important features of the \texttt{ADAQAcquisitionRootGUI} program. Finally,
Chapter~\ref{chap:source} provides an overview of the
\texttt{ADAQAcquisitionRootGUI} source code to enable the ambitious user to add
his/her own improvements to the software.

    
\section{Software Dependencies}
\label{sec: dep}
This section briefly describes the external software dependencies
required to build and execute the \ADAQ source code.

\subsection{The CAEN libraries}
\label{sec: caendep}
CAEN distributes hardware drivers and C libraries in support of its
data acquisition system hardware. The following is a complete list of
the required CAEN libraries for \ADAQ. Note that the most up-to-date
version of the CAEN libraries and driver that \textit{have been
  successfully tested} with \ADAQ are listed in parenthesis next to the
library name:
\begin{enumerate}
\item{\textbf{CAENVMELib} (\textit{CAENVMELib-2.30}): provides a set
    of C functions that enable communication with the VME crate
    backplane and attached VME modules with the CAEN ``bridge'' modules,
    such as the VME-USB V1718 board.}
\item{\textbf{CAENComm} (\textit{CAENComm-1.02}): provides a set of
    relatively high-level C functions that mask the low-level details
    of communication with VME hardware via VME protocol, such as
    reading/writing registers and other common VME functionality.}
\item{\textbf{CAENDigitizer} (\textit{CAENDigitizer-1.31}): provides a
    set of relatively high-level C functions that mask the low-level
    details of communication with the CAEN family of
    digitizers. Explicitly meant for digitial data acquisition with
    CAEN sampling ADC (analog-to-digital) modules such as the V1720
    board.}
\item{\textbf{V1718 Linux Driver} (\textit{Version 0.9}): provides
  necessary Linux driver to interface with the VME-USB V1718
  board. Must be built from source and installed before using
  \texttt{ADAQAcquisitionRootGUI}.}
\end{enumerate}
The \CAEN dependencies listed above are provided as part of the \ADAQ
source code distribution for convenience and to ensure explicit
version of control of the various libraries.\\

\noindent
The \CAEN homepage: \purl{http://www.CAEN.com}\\

\subsection{The \ROOT toolkit}
\label{sec: rootdep}
\ROOT\ is a free (as in no cost), open-source, C++ object-oriented
data analysis toolkit that was originated at CERN but is now developed
and maintained by a worldwide collaboration across many scientific
fields, although it is primarily used in particle and nuclear physics.

While it was developed explicitly for handling massive data sets, it
is also incredibly useful for data analysis on a small scale,
providing powerful histogramming, fitting, graphical output, and GUI
generation. It also contains a diverse array of powerful data analysis
tools that are available as class objects that can be applied to any
data set of data regardless of size. \ROOT is also powerful because
the user can fold his or her own C++ code with the \ROOT class objects
to create highly dynamic and customized data analysis, using all of
the power of C++: the standard libraries, the \BOOST libraries,
etc. It is open-source, constantly improved and maintained, and there
is a dynamic user community available online.\\

\noindent
The \ROOT homepage: \purl{http://root.cern.ch/drupal/}\\

\noindent
The \textsc{ROOTTalk} users forum: \purl{http://root.cern.ch/phpBB2/}\\

\subsection{The \BOOST C++ libraries}
\label{sec: boostdep}
\BOOST is a set of free (as in no cost), open-source, peer-reviewed
portable C++ libraries that are incredibly powerful and diverse.
\BOOST can be thought of as a significant extension of the C++
Standard Libraries and is often a testing ground for a majority of the
code that eventually gets folded into C++ during technical reviews.\\

\noindent
The \BOOST homepage: \purl{http://www.boost.org}\\
