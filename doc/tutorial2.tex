\chapter{Tutorial 2: Saving digitized waveforms to a \ROOT file}
\label{chap:tutorial2}
This chapter provides a detailed, step-by-step tutorial that covers
using \ADAQ to store digitized detector waveforms into a \ROOT file.

The user is expected to have covered Tutorial 1 before beginning this
tutorial, as a number of important skills taught there are critical to
this tutorial as well.

\section{Assumptions}
\label{sec:assumptions2}
The assumptions for this tutorial are identical to those found in
Tutorial 1. Please see Section~\ref{sec:assumptions1}.

\section{Hardware setup}
\label{sec:hardwaresetup2}
The hardware setup for this tutorial is identical to that found in
Tutorial 1. Please see Section~\ref{sec:hardwaresetup1}

\section{Configure \ADAQ for acquisition}
\label{sec:configure2}

\section{Acquiring waveforms}
\label{sec:acquiring2}
Setting up waveform acquisition is identical to that found in Tutorial
1. Please see Section~\ref{sec:acquiring1}.

\section{Saving waveforms  into a \ROOT file}
The section describes the steps to create a \ROOT file, store
digitized waveforms in it, and successfully close the \ROOT file.
At this point in the tutorial

\begin{enumerate}
  \item{Configure all waveform acquisition settings to their desired
    value using the channel 0 specific settings found towards the
    top-right of the Oscilloscope frame. When done adjusting channel
    0's settings, stop the acquisition.}
  \item{Click the ``Data Storage'' subtab toward the bottom right of
    the Oscilloscope frame. Note that all the widgets on this
    subframe are disabled.}
  \item{Start the acquisition. Note that ``Filename'', ``Comment'',
    and ``Create ROOT File'' widgets are enabled, indicating that
    these widgets may only be used when acquisition is running.}
  \item{Set the \ROOT filename. Note that the user is responsible to
    providing an appropriate suffix, with ``.root'' being the
    recommended value. When done, click the ``Create ROOT File''
    button. This creates and initializes the \ROOT file on the hard
    disk, as well as enables the ``Close ROOT file'' and ``Data stored
    ...'' check box, since the \ROOT file is now open and can receive
    data. Note that at this point no waveforms are being written to
    the \ROOT file even though waveforms are being acquired by
    \texttt{|CyDAQRootGUI}.}
  \item{Click the ``Data stored ...'' check box at the bottom of the
    subframe. The digitized waveforms on channel 0 that appear on the
    DGScope canvas are now being written to the \ROOT file. The user
    may continually view the size of the created \ROOT file to
    convince him- or herself that the \ROOT file is receiving data and
    growing in size.}
  \item{Once it is determined that sufficient data has been acquired
    (by viewing and being satisfied with a pulse spectrum or
    completing a 5 minute data acquisition session, for example),
    uncheck the ``Data stored ...'' check box. Waveforms have ceased
    to be written to the \ROOT file; the \ROOT file is still open and
    must be closed correctly to ensure the file is not corrupted.}
  \item{Click the ``Close ROOT File'' button. This performs a number
    of final, required operations on the \ROOT file and closes it.}
\end{enumerate}

At this point in the tutorial, a complete \ROOT file exists on the
hard disk containing digitized waveforms and can be processed with
offline data acquisition code. The user may now: create a new \ROOT
file and repeat the above steps; cease acquisition, modify channel
settings and repeat the above steps; or move on to a completely
different task.

\section{Offline analysis using the \ROOT file}
The \ADAQ user is responsible for developing his or her own code that
will be used offline to process the data contained in the \ROOT
file. To guide the user in creating his or her own analysis code, a
data analysis ``template'' has been created that explicitly shows the
user how to:
\begin{itemize}
  \item{create a standalone, C++ and \ROOT based offline analysis code}
  \item{open a \ADAQ \ROOT file and extract the V6534 and V1720 channel parameters}
  \item{extract the digitized waveforms into C++ integer arrays for processing}
\end{itemize}
The template is configured to open a \ROOT file (the path to the file
must be specified on the command line), print the V6534 and V1720
channel parameters, and the cycle through all of the V1720 channel 0
waveforms contained in the \ROOT file.

The user should copy the template directory to a new directory, and
use it as the basis for his or her own analysis code. All of the
critical interactions with the \ROOT file that must be performed to
extract all stored data are explicitly shown and heavily commented,
providing the user with a good understanding of the steps as well as a
solid foundation for his or her own analysis code. The template code
may be found in the \texttt{\$CYDAQHOME/analysis/analysisTemplate/}
directory.

\section{Conclusion}
Congratulations! If you have reached this point without throwing
errors or cursing the author with reckless abondon, you have
successfully used \ADAQ to acquire detector waveform data and store it
in a \ROOT file for permament storage and offline data analysis. The
interested user is recommended to repeat this tutorial but explore the
numerous \texttt{CyDAQRootGUI} settings that were not used in this
tutorial, especially acquiring data in coincidence and using the level
discrimanator settings as the ``trigger'' for storing waveforms in the
\ROOT file.

