\chapter{Introduction to ADAQAcquisition}
\label{chap:introdurction}

\ADAQ is a program that provides a powerful graphical user interface
(GUI) method for the interacting, programming, and acquiring digitized
waveform data using CAEN high voltage and acquisition hardware. It
provides the user with a full range of functionality, from high level
methods for accomplishing complex tasks all the way to reading/writing
individual registers on the hardware. In addition to a fully featured
digital oscilloscope and multichannel analyzer, the program provides
persistent storage of waveform data in compressed binary files for
offline analysis with its sister program (\texttt{ADAQAnalysis}),
publication ready graphical output of waveforms and spectra, and
control over VME high voltage, VME-USB/OpticalLink bridge, and pulser
boards.

\ADAQ is built upon two main dependencies. The first is the
\texttt{ADAQ} library. The \texttt{ADAQ} library is a custom library
developed to provide standardized, straightforward, and enhanced
control over CAEN hardware. The library is available as a C++ shared
object library and Python module, encapsulating all of the standard
CAEN libraries with wrappers and extending these libraries by adding
new features. The second dependency is the \ROOT toolkit, which
provides object-oriented framework for data analysis, persistent
storage, and GUI construction.

At present, \ADAQ is only supported and tested on Linux platforms,
although in principle, building on Mac should be fairly straightfoward
provided the user's environment is correctly configured.


\section{Purpose of the \ADAQ User's Guide}
\label{sec: overviewsoftware}

The purpose of this user's guide is to provide the user of \ADAQ with
the information required to:
\begin{enumerate}
\item{Retrieve the \ADAQ source code from its public \GIT repository}
\item{Understand the general layout and build system of the \ADAQ
  source code}
\item{Configure the user's environment and build the \ADAQ binary}
\item{Educate the user in all the features of \ADAQ}
\end{enumerate}

Detailed documentation on the \CAEN libraries can be found here:\\
\href{http://www.caen.it/csite/LibrarySearch.jsp}{http://www.caen.it/csite/LibrarySearch.jsp}\\

Documentation on the \ROOT toolkit can be found here:\\
\href{http://root.cern.ch/drupal/content/documentation}{http://root.cern.ch/drupal/content/documentation}\\

To achieve these goals, the remainder of this chapter describes the
external software dependencies required to build the \ADAQ source code
and run its programs. Chapter~\ref{chap:svn} provides instructions on
how to obtain the \ADAQ source from the \GIT repository and building
\texttt{ADAQAcquisitionRootGUI} from source, including configuring the user's
Linux environment and some notes on obtaining and building the
external software dependencies. Chapter~\ref{chap:overview} provides a
detailed guide to the \texttt{ADAQAcquisitionRootGUI} program, including full
textual descriptions of each graphical widget available to the user as
well as annotated graphics of various parts of the
program. Chapters~\ref{chap:tutorial1} and \ref{chap:tutorial2}
contain step-by-step tutorials that walk the user through some of the
most important features of the \texttt{ADAQAcquisitionRootGUI} program. Finally,
Chapter~\ref{chap:source} provides an overview of the
\texttt{ADAQAcquisitionRootGUI} source code to enable the ambitious user to add
his/her own improvements to the software.

    
\section{Software Dependencies}
\label{sec: dep}
This section briefly describes the external software dependencies
required to build and execute the \ADAQ source code.

\subsection{The CAEN libraries}
\label{sec: caendep}
CAEN distributes hardware drivers and C libraries in support of its
data acquisition system hardware. The following is a complete list of
the required CAEN libraries for \ADAQ. Note that the most up-to-date
version of the CAEN libraries and driver that \textit{have been
  successfully tested} with \ADAQ are listed in parenthesis next to the
library name:
\begin{enumerate}
\item{\textbf{CAENVMELib} (\textit{CAENVMELib-2.30}): provides a set
    of C functions that enable communication with the VME crate
    backplane and attached VME modules with the CAEN ``bridge'' modules,
    such as the VME-USB V1718 board.}
\item{\textbf{CAENComm} (\textit{CAENComm-1.02}): provides a set of
    relatively high-level C functions that mask the low-level details
    of communication with VME hardware via VME protocol, such as
    reading/writing registers and other common VME functionality.}
\item{\textbf{CAENDigitizer} (\textit{CAENDigitizer-1.31}): provides a
    set of relatively high-level C functions that mask the low-level
    details of communication with the CAEN family of
    digitizers. Explicitly meant for digitial data acquisition with
    CAEN sampling ADC (analog-to-digital) modules such as the V1720
    board.}
\item{\textbf{V1718 Linux Driver} (\textit{Version 0.9}): provides
  necessary Linux driver to interface with the VME-USB V1718
  board. Must be built from source and installed before using
  \texttt{ADAQAcquisitionRootGUI}.}
\end{enumerate}
The \CAEN dependencies listed above are provided as part of the \ADAQ
source code distribution for convenience and to ensure explicit
version of control of the various libraries.\\

\noindent
The \CAEN homepage: \purl{http://www.CAEN.com}\\

\subsection{The \ROOT toolkit}
\label{sec: rootdep}
\ROOT\ is a free (as in no cost), open-source, C++ object-oriented
data analysis toolkit that was originated at CERN but is now developed
and maintained by a worldwide collaboration across many scientific
fields, although it is primarily used in particle and nuclear physics.

While it was developed explicitly for handling massive data sets, it
is also incredibly useful for data analysis on a small scale,
providing powerful histogramming, fitting, graphical output, and GUI
generation. It also contains a diverse array of powerful data analysis
tools that are available as class objects that can be applied to any
data set of data regardless of size. \ROOT is also powerful because
the user can fold his or her own C++ code with the \ROOT class objects
to create highly dynamic and customized data analysis, using all of
the power of C++: the standard libraries, the \BOOST libraries,
etc. It is open-source, constantly improved and maintained, and there
is a dynamic user community available online.\\

\noindent
The \ROOT homepage: \purl{http://root.cern.ch/drupal/}\\

\noindent
The \textsc{ROOTTalk} users forum: \purl{http://root.cern.ch/phpBB2/}\\

\subsection{The \BOOST C++ libraries}
\label{sec: boostdep}
\BOOST is a set of free (as in no cost), open-source, peer-reviewed
portable C++ libraries that are incredibly powerful and diverse.
\BOOST can be thought of as a significant extension of the C++
Standard Libraries and is often a testing ground for a majority of the
code that eventually gets folded into C++ during technical reviews.\\

\noindent
The \BOOST homepage: \purl{http://www.boost.org}\\
